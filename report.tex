\documentclass{article}

\usepackage[utf8]{inputenc}

\usepackage{geometry}
\geometry{a4paper}

\usepackage{graphicx}
\usepackage{amsmath}

%%% PACKAGES
\usepackage{booktabs} % for much better looking tables
\usepackage{array} % for better arrays (eg matrices) in maths
\usepackage{paralist} % very flexible & customisable lists (eg. enumerate/itemize, etc.)
\usepackage{verbatim} % adds environment for commenting out blocks of text & for better verbatim
\usepackage{subfig} % make it possible to include more than one captioned figure/table in a single float
\usepackage{lastpage}

%%% HEADERS & FOOTERS
\usepackage{fancyhdr} % This should be set AFTER setting up the page geometry
\pagestyle{fancy} % options: empty , plain , fancy
\renewcommand{\headrulewidth}{0pt} % customise the layout...
\lhead{}\chead{}\rhead{}
\lfoot{}\cfoot{\thepage}\rfoot{}

%%% SECTION TITLE APPEARANCE
\usepackage{sectsty}
\allsectionsfont{\sffamily\mdseries\upshape}

%%% ToC (table of contents) APPEARANCE
\usepackage[nottoc,notlof,notlot]{tocbibind} % Put the bibliography in the ToC
\usepackage[titles,subfigure]{tocloft} % Alter the style of the Table of Contents
\renewcommand{\cftsecfont}{\rmfamily\mdseries\upshape}
\renewcommand{\cftsecpagefont}{\rmfamily\mdseries\upshape} % No bold!

\usepackage[backend=biber]{biblatex}
\addbibresource{/home/teo/extra/uit/bib/ref.bib}

%%% END Article customizations

%%%%%%%%%%%%%%%%%%%%%%%%%
%%%          Fill in the title details                      %%%

\def\courseCode{--course-code--}
\def\course{--course--}
\def\thetitle{--title--}
\def\theauthor{--name--}

%%%%%%%%%%%%%%%%%%%%%%%%%

%\pagestyle{fancy}
\pagestyle{fancyplain} % options: empty , plain , fancy
\renewcommand{\headrulewidth}{1pt} % customise the layout...
\renewcommand{\footrulewidth}{0pt}
\lhead{\fancyplain{}{\courseCode: \thetitle{}}}\chead{}\rhead{\fancyplain{}{\theauthor{}}}
\lfoot{}\cfoot{Page {\thepage} of~\pageref{LastPage}}\rfoot{}

\begin{document}

%%% TITLE HEADER

\begin{center}
\textsc{\Large \courseCode: \course}\\[0.5cm]

\textsc{\LARGE \thetitle}\\[1.0cm]

\LARGE{\theauthor} \\[0.5cm]

{\large \today}

\end{center}


%%% DOCUMENT BODY

%%% Set counter to 1
\setcounter{page}{1}

\section{Introduction}

Short introduction to the assignment, motivation and expected results. Remember to change ``Ola Norman'' in the header and front page to your own name.

\subsection{Requirements}

Outline the detailed requirements specified in the assignment text.

\begin{itemize}

\item First requirement
\item Next requirement
\item etc.

\end{itemize}


\section{Technical Background}

Which topics are covered in this assignment. Should be short and cover the necessary topics without mentioning your specific implementation and design.


\section{Design}

How did you solve the assignment? Describe the architecture and any design choices you've made. Show figures of the proposed architecture.

Remember to refer to figures, such as Figure \ref{fig:ackseq} below, in your text.

%% An example figure
%\begin{figure}[h!]
%\begin{center}
%\includegraphics[width=177px] {figureTree.pdf}
%\end{center}
%\caption{A binary search tree}\label{fig:ackseq}
%\end{figure}

If this section gets too small, it can be merged with the Implementation section below with the title ``Design and Implementation''.

\section{Implementation}

How did you implement, deploy and run your application? No need to refer to actual lines of code.

\section{Discussion}

Any advantages or disadvantages with your design? Anything that was surprising to you?

\subsection{Evaluation}

This section should contain relevant graphs and test results, if available/relevant.

\section{Conclusion}

Sum up by restating the problem and solution. Follow up with a brief summary of the solution along with lessons learned. 

Finally, remember to use all references you list\cite{pathen98}.


%%% BIBLOGRAPHY

\newpage{}

\printbibliography

\end{document}
